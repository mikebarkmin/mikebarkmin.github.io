% LTeX: language=de-DE
\documentclass[a4paper,14pt]{scrartcl}
\usepackage[utf8]{luainputenc}
\usepackage[
	typ=ab,
	fach=Informatik,
	lerngruppe=6,
  zitate=quotes,
  namensfeldAnzeigen,
  datumAnzeigen,
	lizenz=cc-by-sa-4,
	farbig,
	module={Symbole,Lizenzen}
]{schule}

\usepackage[skip=10pt plus1pt]{parskip}
\usepackage{emoji}
\renewcommand{\baselinestretch}{1.5} 

\ifoot{v1.0.0}
\cfoot{}
\ofoot{\lizenzSymbol}

\usepackage[left=2cm,right=2cm,top=1cm,bottom=1cm,includehead,includefoot]{geometry}

\title{Speichermedien}
\author{Mike Barkmin}
\date{Datum: \hspace{1.5cm}}

\begin{document}
\section*{\emoji{green-circle} USB-Stick}

Ein USB-Stick ist ein kleines, tragbares Speichermedium, das du in den USB-Anschluss eines Computers, Laptops oder anderer Geräte stecken kannst. 

Er wird oft verwendet, um Dateien wie Dokumente, Fotos oder Musik von einem Gerät auf ein anderes zu übertragen. 

USB-Sticks funktionieren mit fast allen Computern und Laptops sowie einigen Smart-TVs und Spielekonsolen. Es gibt auch spezielle USB-Sticks für Smartphones und Tablets mit passenden Anschlüssen. 

USB-Sticks gibt es in verschiedenen Größen, von wenigen Gigabyte (GB) bis zu mehreren Hundert Gigabyte. 

Sie sind günstig, und je mehr Speicherplatz sie haben, desto teurer werden sie.

Ein USB-Stick funktioniert, indem er Daten in einem Speicherchip speichert, ähnlich wie eine SD-Karte. Er hat keine beweglichen Teile und ist sehr einfach zu benutzen, aber er kann nach vielen Nutzungen kaputtgehen.

\end{document}
