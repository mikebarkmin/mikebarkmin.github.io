% LTeX: language=de-DE
\documentclass[a4paper,14pt]{scrartcl}
\usepackage[utf8]{luainputenc}
\usepackage[
	typ=ab,
	fach=Informatik,
	lerngruppe=6,
  zitate=quotes,
  namensfeldAnzeigen,
  datumAnzeigen,
	lizenz=cc-by-sa-4,
	farbig,
	module={Symbole,Lizenzen}
]{schule}

\usepackage[skip=10pt plus1pt]{parskip}
\usepackage{emoji}
\renewcommand{\baselinestretch}{1.5} 

\ifoot{v1.0.0}
\cfoot{}
\ofoot{\lizenzSymbol}

\usepackage[left=2cm,right=2cm,top=1cm,bottom=1cm,includehead,includefoot]{geometry}

\title{Speichermedien}
\author{Mike Barkmin}
\date{Datum: \hspace{1.5cm}}

\begin{document}
\section*{\emoji{orange-circle} Hard Disk Drive (HDD)}

Eine HDD (Hard Disk Drive) ist eine Festplatte, die du oft in Desktop-Computern und Laptops findest, um viele Daten zu speichern. 

Du kannst eine HDD verwenden, um Programme, Spiele, Filme und viele andere Dateien zu speichern. 

Sie wird auch in Servern und Speichersystemen großer Unternehmen verwendet, um riesige Datenmengen zu speichern. 

Festplatten haben eine große Kapazität, oft mehrere Terabyte (1 Terabyte = 1000 Gigabyte), und sie sind relativ günstig.

Eine HDD funktioniert, indem sie Daten auf rotierenden Scheiben speichert. Ein kleiner Arm bewegt sich über diese Scheiben, um Daten zu lesen oder zu schreiben. Da sie bewegliche Teile hat, kann sie kaputtgehen, wenn sie oft benutzt oder fallen gelassen wird.

\end{document}
