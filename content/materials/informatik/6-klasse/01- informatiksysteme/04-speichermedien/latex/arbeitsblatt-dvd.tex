% LTeX: language=de-DE
\documentclass[a4paper,14pt]{scrartcl}
\usepackage[utf8]{luainputenc}
\usepackage[
	typ=ab,
	fach=Informatik,
	lerngruppe=6,
  zitate=quotes,
  namensfeldAnzeigen,
  datumAnzeigen,
	lizenz=cc-by-sa-4,
	farbig,
	module={Symbole,Lizenzen}
]{schule}

\usepackage[skip=10pt plus1pt]{parskip}
\usepackage{emoji}
\renewcommand{\baselinestretch}{1.5} 

\ifoot{v1.0.0}
\cfoot{}
\ofoot{\lizenzSymbol}

\usepackage[left=2cm,right=2cm,top=1cm,bottom=1cm,includehead,includefoot]{geometry}

\title{Speichermedien}
\author{Mike Barkmin}
\date{Datum: \hspace{1.5cm}}

\begin{document}
\section*{\emoji{orange-circle} DVD}

Eine DVD ist eine Scheibe, die du in ein DVD-Laufwerk von Computern, speziellen DVD-Playern oder Spielekonsolen legen kannst, um Daten wie Filme, Musik oder Dokumente zu speichern. 

DVDs können etwa 4,7 Gigabyte (GB) an Daten speichern, was genug für einen Film ist. 

Sie werden oft in Computern und Laptops verwendet, um Software zu installieren oder Daten zu speichern. 

DVDs sind sehr günstig und kosten oft nur wenige Euro.

Eine DVD speichert Daten, indem ein Laser kleine Vertiefungen auf der Scheibe macht. Ein DVD-Player, ein Computer oder eine Spielekonsole liest diese Vertiefungen, um die Daten wiederzugeben. Eine DVD kann zerkratzen und dann nicht mehr funktionieren, deshalb sollte man sie vorsichtig behandeln.

\end{document}
