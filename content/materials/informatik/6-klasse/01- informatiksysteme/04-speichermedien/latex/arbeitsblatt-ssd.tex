% LTeX: language=de-DE
\documentclass[a4paper,14pt]{scrartcl}
\usepackage[utf8]{luainputenc}
\usepackage[
	typ=ab,
	fach=Informatik,
	lerngruppe=6,
  zitate=quotes,
  namensfeldAnzeigen,
  datumAnzeigen,
	lizenz=cc-by-sa-4,
	farbig,
	module={Symbole,Lizenzen}
]{schule}

\usepackage[skip=10pt plus1pt]{parskip}
\usepackage{emoji}
\renewcommand{\baselinestretch}{1.5} 

\ifoot{v1.0.0}
\cfoot{}
\ofoot{\lizenzSymbol}

\usepackage[left=2cm,right=2cm,top=1cm,bottom=1cm,includehead,includefoot]{geometry}

\title{Speichermedien}
\author{Mike Barkmin}
\date{Datum: \hspace{1.5cm}}

\begin{document}
\section*{\emoji{orange-circle} Solid State Drive (SSD)}

Eine SSD ist ein moderner Speicher, der viel schneller ist als eine HDD. 

Du kannst sie verwenden, um Programme, Spiele und große Dateien auf deinem Computer oder Laptop zu speichern. 

SSDs werden in Laptops, Desktop-Computern und immer häufiger auch in Tablets verwendet. Sie werden auch in leistungsstarken Servern und Spielkonsolen eingesetzt, um schnelles Laden von Daten zu ermöglichen. 

SSDs gibt es in verschiedenen Größen, von einigen hundert Gigabyte bis zu mehreren Terabyte. Sie sind teurer als HDDs, aber auch schneller und langlebiger.

Eine SSD funktioniert ohne bewegliche Teile. Sie speichert Daten in elektronischen Speicherchips, ähnlich wie ein USB-Stick oder eine SD-Karte. Dadurch ist sie schneller und weniger anfällig für Defekte.

\end{document}
