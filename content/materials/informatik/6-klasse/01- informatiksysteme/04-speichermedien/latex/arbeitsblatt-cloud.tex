% LTeX: language=de-DE
\documentclass[a4paper,14pt]{scrartcl}
\usepackage[utf8]{luainputenc}
\usepackage[
	typ=ab,
	fach=Informatik,
	lerngruppe=6,
  zitate=quotes,
  namensfeldAnzeigen,
  datumAnzeigen,
	lizenz=cc-by-sa-4,
	farbig,
	module={Symbole,Lizenzen}
]{schule}

\usepackage{emoji}
\usepackage[skip=10pt plus1pt]{parskip}
\renewcommand{\baselinestretch}{1.5} 

\ifoot{v1.0.0}
\cfoot{}
\ofoot{\lizenzSymbol}

\usepackage[left=2cm,right=2cm,top=1cm,bottom=1cm,includehead,includefoot]{geometry}

\title{Speichermedien}
\author{Mike Barkmin}
\date{Datum: \hspace{1.5cm}}

\begin{document}
\section*{\emoji{red-square} Cloud}

Cloud-Speicher ist eine Möglichkeit, Daten über das Internet zu speichern. Anstatt Daten auf einem Gerät wie einem USB-Stick oder einer Festplatte zu speichern, speicherst du sie auf Servern, die von Firmen wie Google, Apple oder Microsoft bereitgestellt werden. 

Du kannst von überall auf deine Daten zugreifen, solange du eine Internetverbindung hast. 

Cloud-Speicher wird oft mit Computern, Tablets und Smartphones verwendet, um Daten wie Dokumente, Fotos oder Videos zu speichern. 

Viele moderne Programme, besonders in Unternehmen, greifen auf die Cloud zurück, um Daten gemeinsam zu nutzen und zu speichern. 

Die Cloud kann sehr viel Daten speichern, und oft bekommst du ein paar Gigabyte kostenlos. Für mehr Speicherplatz musst du monatlich zahlen.

Die Cloud funktioniert, indem du deine Dateien über das Internet hochlädst. Die Daten werden auf entfernten Servern gespeichert, die du nicht sehen oder anfassen kannst. Ein Vorteil der Cloud ist, dass du von jedem Gerät mit Internetzugang auf deine Daten zugreifen kannst. Der Nachteil ist, dass du ohne Internet keine Daten speichern oder abrufen kannst.

\end{document}
