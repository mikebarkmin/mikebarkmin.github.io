% LTeX: language=de-DE
\documentclass[a4paper,14pt]{scrartcl}
\usepackage[utf8]{luainputenc}
\usepackage[
	typ=ab,
	fach=Informatik,
	lerngruppe=6,
  zitate=quotes,
  namensfeldAnzeigen,
  datumAnzeigen,
	lizenz=cc-by-sa-4,
	farbig,
	module={Symbole,Lizenzen}
]{schule}

\usepackage[skip=10pt plus1pt]{parskip}
\usepackage{emoji}
\renewcommand{\baselinestretch}{1.5} 

\ifoot{v1.0.0}
\cfoot{}
\ofoot{\lizenzSymbol}

\usepackage[left=2cm,right=2cm,top=1cm,bottom=1cm,includehead,includefoot]{geometry}

\title{Speichermedien}
\author{Mike Barkmin}
\date{Datum: \hspace{1.5cm}}

\begin{document}
\section*{\emoji{green-circle} SD-Karte}

Eine SD-Karte ist ein kleiner Speicherchip, der in vielen Geräten wie Kameras, Handys, Tablets und Computern verwendet wird. 


Du kannst sie einfach in den passenden Schlitz des Geräts stecken, um Fotos, Videos oder Musik zu speichern. 

SD-Karten werden oft in Smartphones und Digitalkameras genutzt, um Bilder und Videos zu speichern. Sie sind auch in Laptops oder Spielekonsolen zu finden, um zusätzlichen Speicherplatz zu bieten. 

SD-Karten können unterschiedlich viel Daten speichern, von wenigen Gigabyte (GB) bis zu mehreren Hundert Gigabyte. 

Eine SD-Karte ist günstig; kleine Karten kosten nur wenige Euro, größere Karten sind teurer.

Eine SD-Karte funktioniert, indem sie Daten elektronisch speichert, ähnlich wie ein USB-Stick. Sie hat keine beweglichen Teile, was sie robuster macht, aber sie kann mit der Zeit kaputtgehen, wenn sie oft benutzt wird.

\end{document}
