% LTeX: language=de-DE
\documentclass[a4paper]{scrartcl}
\usepackage[
	typ=ab,
	fach=Informatik,
	lerngruppe=EF,
  zitate=quotes,
  namensfeldAnzeigen,
  datumAnzeigen,
	lizenz=cc-by-sa-4,
	farbig,
	module={Symbole, Lizenzen}
]{schule}
\usepackage{qrcode}

\ifoot{v2024-02-19}
\cfoot{}
\ofoot{\lizenzSymbol}

\title{Evolutionäre Algorithmen}
\author{Mike Barkmin}
\date{Datum: \hspace{1.5cm}}

\begin{document}
\section*{Evolutionäre Algorithmen}

Du wirst im Folgenden eine Kreatur erschaffen, die eigenständig lernt sich fortzubewegen. Dazu wird im Hintergrund ein evolutionärer Algorithmus verwendet. Manche deine Kreaturen werden es nicht weit schaffen, andere hingegen werden über den Bildschirm flitzen.

\begin{aufgabe}[subtitle=Einführung]
Damit du verstehst was dich genau erwartet, schau dir das folgende Video an:


\url{https://www.youtube.com/watch?v=A13u9QcoYRM}

\qrcode{https://www.youtube.com/watch?v=A13u9QcoYRM}.

\end{aufgabe}

\begin{aufgabe}[subtitle=Evolution beobachten]

  Zuerst sollst du verstehen wie das Tool funktioniert. Dazu sollst du einer Kreatur zuschauen wie sie sich entwickelt.


  \url{https://keiwan.itch.io/evolution}

  \qrcode{https://keiwan.itch.io/evolution}


  \begin{teilaufgaben}
    \teilaufgabe Durchlaufe das Tutorial.
    \teilaufgabe Klicke auf Unnamed und lade die Kreatur \enquote{Frogger}.
    \teilaufgabe Klicke auf \enquote{Evolve}.
    \teilaufgabe Warte eine Zeit und schaue der Evolution zu.
  \end{teilaufgaben}


\end{aufgabe}

\begin{aufgabe}[subtitle=Hintergrund]
  Lies dir den Text auf der nächsten Seite zu evolutionären Algorithmen durch, um zu verstehen, wie sich deine Kreatur immer weiterentwickelt.
\end{aufgabe}

\begin{aufgabe}[subtitle=Eigene Kreatur]
  Jetzt bist du an der Reihe. Erstelle eine eigene Kreatur.

  Falls du mehr Inspiration brauchst, kannst du dir diese Galerie von Kreaturen anschauen.

  \url{https://keiwando.com/evolution/gifs/}

  \qrcode{https://keiwando.com/evolution/gifs/}
\end{aufgabe}

\newpage

\section*{Was ist ein evolutionärer Algorithmus?}

Ein evolutionärer Algorithmus ist eine Art von Algorithmus für maschinelles Lernen, der vom biologischen Evolutionsprozess inspiriert ist.


Das Ziel von Algorithmen für maschinelles Lernen besteht im Allgemeinen darin, eine optimale Lösung für ein Optimierungsproblem zu finden. Im Fall dieses Simulators wird das Ziel durch die Aufgabe vorgegeben, die du für deine Kreatur ausgewählt hast (z. B. \enquote{so schnell wie möglich laufen}).


Je besser eine Kreatur diese Aufgabe bewältigt, desto höher ist ihr Fitnesswert. Dieser Fitnesswert ist das, was hier optimiert wird. Der Algorithmus versucht, ein Verhalten des Lebewesens zu finden, das den höchstmöglichen Fitnesswert erzeugt.


Evolutionäre Algorithmen versuchen, solche optimalen Lösungen durch den Prozess von Versuch und Irrtum zu finden. Sie probieren wiederholt eine Reihe von Lösungskandidaten aus, bewerten sie und generieren dann auf der Grundlage dieser Bewertung eine neue Reihe möglicher Lösungen. (Eine \enquote{Lösung} entspricht in diesem Artikel dem Verhalten eines einzelnen Lebewesens, das durch die Gewichte seines Gehirns bzw. seines neuronalen Netzes definiert wird).


Im Folgenden wird der allgemeine Aufbau eines evolutionären Algorithmus beschrieben:

\begin{enumerate}
  \item Erstellen einer zufälligen Menge von Ausgangslösungen (einer Population)

  Wiederhole die Schritte 2. - 5. -
  \item Bewerte jede Lösung danach, wie gut sie das Problem löst
  \item Auswahl der Elternlösungen, die ihre Eigenschaften an die nächste Generation weitergeben dürfen
  \item Erstellen einer neuen Gruppe von Lösungen für die nächste Iteration durch Rekombination von Elternlösungen
  \item Hinzufügen zufälliger Änderungen (Mutation) zu einigen der neuen Lösungen
\end{enumerate}


Die Mutationen in Schritt 5 zielen darauf ab, die Wahrscheinlichkeit zu verringern, dass der Algorithmus in einem sogenannten lokalen Optimum stecken bleibt. Ein lokales Optimum der Fitnessfunktion ist durch die Fitness einer Lösung gegeben, die sich verschlechtert, wenn man sie in irgendeiner Weise leicht verändert. Das Problem dabei ist, dass eine solche (lokale) Lösung nicht unbedingt die globale Lösung ist, nach der man eigentlich sucht.


Die Hinzufügung eines völlig zufälligen Faktors in jeder Generation kann ausreichen, um neue Lösungen zu erzeugen, die sich so stark von einer lokalen Lösung unterscheiden, dass sie die Population davon abbringen können, in diesem lokalen Optimum stecken zu bleiben.

\end{document}
